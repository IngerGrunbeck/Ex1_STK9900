% Options for packages loaded elsewhere
\PassOptionsToPackage{unicode}{hyperref}
\PassOptionsToPackage{hyphens}{url}
%
\documentclass[
]{article}
\usepackage{amsmath,amssymb}
\usepackage{lmodern}
\usepackage{iftex}
\ifPDFTeX
  \usepackage[T1]{fontenc}
  \usepackage[utf8]{inputenc}
  \usepackage{textcomp} % provide euro and other symbols
\else % if luatex or xetex
  \usepackage{unicode-math}
  \defaultfontfeatures{Scale=MatchLowercase}
  \defaultfontfeatures[\rmfamily]{Ligatures=TeX,Scale=1}
\fi
% Use upquote if available, for straight quotes in verbatim environments
\IfFileExists{upquote.sty}{\usepackage{upquote}}{}
\IfFileExists{microtype.sty}{% use microtype if available
  \usepackage[]{microtype}
  \UseMicrotypeSet[protrusion]{basicmath} % disable protrusion for tt fonts
}{}
\makeatletter
\@ifundefined{KOMAClassName}{% if non-KOMA class
  \IfFileExists{parskip.sty}{%
    \usepackage{parskip}
  }{% else
    \setlength{\parindent}{0pt}
    \setlength{\parskip}{6pt plus 2pt minus 1pt}}
}{% if KOMA class
  \KOMAoptions{parskip=half}}
\makeatother
\usepackage{xcolor}
\usepackage[margin=1in]{geometry}
\usepackage{color}
\usepackage{fancyvrb}
\newcommand{\VerbBar}{|}
\newcommand{\VERB}{\Verb[commandchars=\\\{\}]}
\DefineVerbatimEnvironment{Highlighting}{Verbatim}{commandchars=\\\{\}}
% Add ',fontsize=\small' for more characters per line
\usepackage{framed}
\definecolor{shadecolor}{RGB}{248,248,248}
\newenvironment{Shaded}{\begin{snugshade}}{\end{snugshade}}
\newcommand{\AlertTok}[1]{\textcolor[rgb]{0.94,0.16,0.16}{#1}}
\newcommand{\AnnotationTok}[1]{\textcolor[rgb]{0.56,0.35,0.01}{\textbf{\textit{#1}}}}
\newcommand{\AttributeTok}[1]{\textcolor[rgb]{0.77,0.63,0.00}{#1}}
\newcommand{\BaseNTok}[1]{\textcolor[rgb]{0.00,0.00,0.81}{#1}}
\newcommand{\BuiltInTok}[1]{#1}
\newcommand{\CharTok}[1]{\textcolor[rgb]{0.31,0.60,0.02}{#1}}
\newcommand{\CommentTok}[1]{\textcolor[rgb]{0.56,0.35,0.01}{\textit{#1}}}
\newcommand{\CommentVarTok}[1]{\textcolor[rgb]{0.56,0.35,0.01}{\textbf{\textit{#1}}}}
\newcommand{\ConstantTok}[1]{\textcolor[rgb]{0.00,0.00,0.00}{#1}}
\newcommand{\ControlFlowTok}[1]{\textcolor[rgb]{0.13,0.29,0.53}{\textbf{#1}}}
\newcommand{\DataTypeTok}[1]{\textcolor[rgb]{0.13,0.29,0.53}{#1}}
\newcommand{\DecValTok}[1]{\textcolor[rgb]{0.00,0.00,0.81}{#1}}
\newcommand{\DocumentationTok}[1]{\textcolor[rgb]{0.56,0.35,0.01}{\textbf{\textit{#1}}}}
\newcommand{\ErrorTok}[1]{\textcolor[rgb]{0.64,0.00,0.00}{\textbf{#1}}}
\newcommand{\ExtensionTok}[1]{#1}
\newcommand{\FloatTok}[1]{\textcolor[rgb]{0.00,0.00,0.81}{#1}}
\newcommand{\FunctionTok}[1]{\textcolor[rgb]{0.00,0.00,0.00}{#1}}
\newcommand{\ImportTok}[1]{#1}
\newcommand{\InformationTok}[1]{\textcolor[rgb]{0.56,0.35,0.01}{\textbf{\textit{#1}}}}
\newcommand{\KeywordTok}[1]{\textcolor[rgb]{0.13,0.29,0.53}{\textbf{#1}}}
\newcommand{\NormalTok}[1]{#1}
\newcommand{\OperatorTok}[1]{\textcolor[rgb]{0.81,0.36,0.00}{\textbf{#1}}}
\newcommand{\OtherTok}[1]{\textcolor[rgb]{0.56,0.35,0.01}{#1}}
\newcommand{\PreprocessorTok}[1]{\textcolor[rgb]{0.56,0.35,0.01}{\textit{#1}}}
\newcommand{\RegionMarkerTok}[1]{#1}
\newcommand{\SpecialCharTok}[1]{\textcolor[rgb]{0.00,0.00,0.00}{#1}}
\newcommand{\SpecialStringTok}[1]{\textcolor[rgb]{0.31,0.60,0.02}{#1}}
\newcommand{\StringTok}[1]{\textcolor[rgb]{0.31,0.60,0.02}{#1}}
\newcommand{\VariableTok}[1]{\textcolor[rgb]{0.00,0.00,0.00}{#1}}
\newcommand{\VerbatimStringTok}[1]{\textcolor[rgb]{0.31,0.60,0.02}{#1}}
\newcommand{\WarningTok}[1]{\textcolor[rgb]{0.56,0.35,0.01}{\textbf{\textit{#1}}}}
\usepackage{graphicx}
\makeatletter
\def\maxwidth{\ifdim\Gin@nat@width>\linewidth\linewidth\else\Gin@nat@width\fi}
\def\maxheight{\ifdim\Gin@nat@height>\textheight\textheight\else\Gin@nat@height\fi}
\makeatother
% Scale images if necessary, so that they will not overflow the page
% margins by default, and it is still possible to overwrite the defaults
% using explicit options in \includegraphics[width, height, ...]{}
\setkeys{Gin}{width=\maxwidth,height=\maxheight,keepaspectratio}
% Set default figure placement to htbp
\makeatletter
\def\fps@figure{htbp}
\makeatother
\setlength{\emergencystretch}{3em} % prevent overfull lines
\providecommand{\tightlist}{%
  \setlength{\itemsep}{0pt}\setlength{\parskip}{0pt}}
\setcounter{secnumdepth}{-\maxdimen} % remove section numbering
\ifLuaTeX
  \usepackage{selnolig}  % disable illegal ligatures
\fi
\IfFileExists{bookmark.sty}{\usepackage{bookmark}}{\usepackage{hyperref}}
\IfFileExists{xurl.sty}{\usepackage{xurl}}{} % add URL line breaks if available
\urlstyle{same} % disable monospaced font for URLs
\hypersetup{
  pdftitle={Mandatory Assignment 1 - STK9900},
  pdfauthor={Inger Grünbeck},
  hidelinks,
  pdfcreator={LaTeX via pandoc}}

\title{Mandatory Assignment 1 - STK9900}
\author{Inger Grünbeck}
\date{2023-03-12}

\begin{document}
\maketitle

Importing libraries:

\begin{Shaded}
\begin{Highlighting}[]
\NormalTok{message }\OtherTok{=} \ConstantTok{FALSE}
\NormalTok{warning }\OtherTok{=} \ConstantTok{FALSE}

\FunctionTok{library}\NormalTok{(rcompanion)}
\FunctionTok{library}\NormalTok{(car)}
\end{Highlighting}
\end{Shaded}

\begin{verbatim}
## Lade nötiges Paket: carData
\end{verbatim}

\begin{Shaded}
\begin{Highlighting}[]
\FunctionTok{library}\NormalTok{(data.table)}
\FunctionTok{library}\NormalTok{(ggplot2)}
\FunctionTok{theme\_set}\NormalTok{(}
  \FunctionTok{theme\_classic}\NormalTok{() }\SpecialCharTok{+} 
    \FunctionTok{theme}\NormalTok{(}\AttributeTok{legend.position =} \StringTok{"top"}\NormalTok{)}
\NormalTok{  )}
\FunctionTok{library}\NormalTok{(MASS)}
\CommentTok{\#library(tidyverse)}
\end{Highlighting}
\end{Shaded}

\hypertarget{exercise-1}{%
\section{Exercise 1}\label{exercise-1}}

Reading in the pollution dataset:

\begin{Shaded}
\begin{Highlighting}[]
\NormalTok{pollution}\OtherTok{=}\FunctionTok{read.table}\NormalTok{(}\StringTok{"https://www.uio.no/studier/emner/matnat/math/STK4900/data/no2.txt"}\NormalTok{, }\AttributeTok{header=}\ConstantTok{TRUE}\NormalTok{)}
\end{Highlighting}
\end{Shaded}

\hypertarget{a}{%
\subsection{a)}\label{a}}

\hypertarget{method}{%
\subsubsection{Method:}\label{method}}

We inspect the mean, median, standard deviation (sd), the IQR, and the
confidence interval in order to analyse variable log.no2. Further a
histogram and boxplot of the variable are printed.

\begin{Shaded}
\begin{Highlighting}[]
\FunctionTok{summary}\NormalTok{(pollution}\SpecialCharTok{$}\NormalTok{log.no2)}
\end{Highlighting}
\end{Shaded}

\begin{verbatim}
##    Min. 1st Qu.  Median    Mean 3rd Qu.    Max. 
##   1.224   3.214   3.848   3.698   4.217   6.395
\end{verbatim}

\begin{Shaded}
\begin{Highlighting}[]
\FunctionTok{print}\NormalTok{(}\StringTok{"SD:"}\NormalTok{)}
\end{Highlighting}
\end{Shaded}

\begin{verbatim}
## [1] "SD:"
\end{verbatim}

\begin{Shaded}
\begin{Highlighting}[]
\FunctionTok{sd}\NormalTok{(pollution}\SpecialCharTok{$}\NormalTok{log.no2)}
\end{Highlighting}
\end{Shaded}

\begin{verbatim}
## [1] 0.7505966
\end{verbatim}

\begin{Shaded}
\begin{Highlighting}[]
\FunctionTok{print}\NormalTok{(}\StringTok{"IQR:"}\NormalTok{)}
\end{Highlighting}
\end{Shaded}

\begin{verbatim}
## [1] "IQR:"
\end{verbatim}

\begin{Shaded}
\begin{Highlighting}[]
\FunctionTok{IQR}\NormalTok{(pollution}\SpecialCharTok{$}\NormalTok{log.no2)}
\end{Highlighting}
\end{Shaded}

\begin{verbatim}
## [1] 1.003067
\end{verbatim}

\begin{Shaded}
\begin{Highlighting}[]
\FunctionTok{print}\NormalTok{(}\StringTok{"The Cofidence Interval"}\NormalTok{)}
\end{Highlighting}
\end{Shaded}

\begin{verbatim}
## [1] "The Cofidence Interval"
\end{verbatim}

\begin{Shaded}
\begin{Highlighting}[]
\FunctionTok{print}\NormalTok{(}\StringTok{"lower limit:"}\NormalTok{)}
\end{Highlighting}
\end{Shaded}

\begin{verbatim}
## [1] "lower limit:"
\end{verbatim}

\begin{Shaded}
\begin{Highlighting}[]
\FunctionTok{mean}\NormalTok{(pollution}\SpecialCharTok{$}\NormalTok{log.no2) }\SpecialCharTok{{-}} \FunctionTok{qt}\NormalTok{(}\FloatTok{0.975}\NormalTok{,}\DecValTok{499}\NormalTok{)}\SpecialCharTok{*}\NormalTok{(}\FunctionTok{sd}\NormalTok{(pollution}\SpecialCharTok{$}\NormalTok{log.no2)}\SpecialCharTok{/}\FunctionTok{sqrt}\NormalTok{(}\DecValTok{500}\NormalTok{))      }\CommentTok{\# lower limit}
\end{Highlighting}
\end{Shaded}

\begin{verbatim}
## [1] 3.632416
\end{verbatim}

\begin{Shaded}
\begin{Highlighting}[]
\FunctionTok{print}\NormalTok{(}\StringTok{"upper limit"}\NormalTok{)}
\end{Highlighting}
\end{Shaded}

\begin{verbatim}
## [1] "upper limit"
\end{verbatim}

\begin{Shaded}
\begin{Highlighting}[]
\FunctionTok{mean}\NormalTok{(pollution}\SpecialCharTok{$}\NormalTok{log.no2) }\SpecialCharTok{+} \FunctionTok{qt}\NormalTok{(}\FloatTok{0.975}\NormalTok{,}\DecValTok{499}\NormalTok{)}\SpecialCharTok{*}\NormalTok{(}\FunctionTok{sd}\NormalTok{(pollution}\SpecialCharTok{$}\NormalTok{log.no2)}\SpecialCharTok{/}\FunctionTok{sqrt}\NormalTok{(}\DecValTok{500}\NormalTok{))     }\CommentTok{\# upper limit}
\end{Highlighting}
\end{Shaded}

\begin{verbatim}
## [1] 3.764319
\end{verbatim}

\begin{Shaded}
\begin{Highlighting}[]
\FunctionTok{plotNormalHistogram}\NormalTok{(pollution}\SpecialCharTok{$}\NormalTok{log.no2, }\AttributeTok{main=}\StringTok{"Histogram of log.no2 with normal distribution"}\NormalTok{)}
\end{Highlighting}
\end{Shaded}

\includegraphics{mandatory_assignment1_files/figure-latex/unnamed-chunk-5-1.pdf}

\begin{Shaded}
\begin{Highlighting}[]
\FunctionTok{boxplot}\NormalTok{(pollution}\SpecialCharTok{$}\NormalTok{log.no2)}
\end{Highlighting}
\end{Shaded}

\includegraphics{mandatory_assignment1_files/figure-latex/unnamed-chunk-5-2.pdf}

We repeat this for the variable pollution\$log.cars:

\begin{Shaded}
\begin{Highlighting}[]
\FunctionTok{summary}\NormalTok{(pollution}\SpecialCharTok{$}\NormalTok{log.cars)}
\end{Highlighting}
\end{Shaded}

\begin{verbatim}
##    Min. 1st Qu.  Median    Mean 3rd Qu.    Max. 
##   4.127   6.176   7.425   6.973   7.793   8.349
\end{verbatim}

\begin{Shaded}
\begin{Highlighting}[]
\FunctionTok{print}\NormalTok{(}\StringTok{"SD:"}\NormalTok{)}
\end{Highlighting}
\end{Shaded}

\begin{verbatim}
## [1] "SD:"
\end{verbatim}

\begin{Shaded}
\begin{Highlighting}[]
\FunctionTok{sd}\NormalTok{(pollution}\SpecialCharTok{$}\NormalTok{log.cars)}
\end{Highlighting}
\end{Shaded}

\begin{verbatim}
## [1] 1.087166
\end{verbatim}

\begin{Shaded}
\begin{Highlighting}[]
\FunctionTok{print}\NormalTok{(}\StringTok{"IQR:"}\NormalTok{)}
\end{Highlighting}
\end{Shaded}

\begin{verbatim}
## [1] "IQR:"
\end{verbatim}

\begin{Shaded}
\begin{Highlighting}[]
\FunctionTok{IQR}\NormalTok{(pollution}\SpecialCharTok{$}\NormalTok{log.cars)}
\end{Highlighting}
\end{Shaded}

\begin{verbatim}
## [1] 1.617332
\end{verbatim}

\begin{Shaded}
\begin{Highlighting}[]
\FunctionTok{print}\NormalTok{(}\StringTok{"The Cofidence Interval"}\NormalTok{)}
\end{Highlighting}
\end{Shaded}

\begin{verbatim}
## [1] "The Cofidence Interval"
\end{verbatim}

\begin{Shaded}
\begin{Highlighting}[]
\FunctionTok{print}\NormalTok{(}\StringTok{"lower limit:"}\NormalTok{)}
\end{Highlighting}
\end{Shaded}

\begin{verbatim}
## [1] "lower limit:"
\end{verbatim}

\begin{Shaded}
\begin{Highlighting}[]
\FunctionTok{mean}\NormalTok{(pollution}\SpecialCharTok{$}\NormalTok{log.cars) }\SpecialCharTok{{-}} \FunctionTok{qt}\NormalTok{(}\FloatTok{0.975}\NormalTok{,}\DecValTok{499}\NormalTok{)}\SpecialCharTok{*}\NormalTok{(}\FunctionTok{sd}\NormalTok{(pollution}\SpecialCharTok{$}\NormalTok{log.cars)}\SpecialCharTok{/}\FunctionTok{sqrt}\NormalTok{(}\DecValTok{500}\NormalTok{))      }\CommentTok{\# lower limit}
\end{Highlighting}
\end{Shaded}

\begin{verbatim}
## [1] 6.877818
\end{verbatim}

\begin{Shaded}
\begin{Highlighting}[]
\FunctionTok{print}\NormalTok{(}\StringTok{"upper limit"}\NormalTok{)}
\end{Highlighting}
\end{Shaded}

\begin{verbatim}
## [1] "upper limit"
\end{verbatim}

\begin{Shaded}
\begin{Highlighting}[]
\FunctionTok{mean}\NormalTok{(pollution}\SpecialCharTok{$}\NormalTok{log.cars) }\SpecialCharTok{+} \FunctionTok{qt}\NormalTok{(}\FloatTok{0.975}\NormalTok{,}\DecValTok{499}\NormalTok{)}\SpecialCharTok{*}\NormalTok{(}\FunctionTok{sd}\NormalTok{(pollution}\SpecialCharTok{$}\NormalTok{log.cars)}\SpecialCharTok{/}\FunctionTok{sqrt}\NormalTok{(}\DecValTok{500}\NormalTok{))     }\CommentTok{\# upper limit}
\end{Highlighting}
\end{Shaded}

\begin{verbatim}
## [1] 7.068866
\end{verbatim}

\begin{Shaded}
\begin{Highlighting}[]
\FunctionTok{plotNormalHistogram}\NormalTok{(pollution}\SpecialCharTok{$}\NormalTok{log.cars, }\AttributeTok{main=}\StringTok{"Histogram of log.cars with normal distribution"}\NormalTok{)}
\end{Highlighting}
\end{Shaded}

\includegraphics{mandatory_assignment1_files/figure-latex/unnamed-chunk-8-1.pdf}

\begin{Shaded}
\begin{Highlighting}[]
\FunctionTok{boxplot}\NormalTok{(pollution}\SpecialCharTok{$}\NormalTok{log.cars)}
\end{Highlighting}
\end{Shaded}

\includegraphics{mandatory_assignment1_files/figure-latex/unnamed-chunk-8-2.pdf}
Finally, a scatterplot of the two variables is plotted in order to
examine the correlation between the pollution level and the number of
cars per hour. Also, the correlation of the variables is calculated:

\begin{Shaded}
\begin{Highlighting}[]
\FunctionTok{plot}\NormalTok{(pollution}\SpecialCharTok{$}\NormalTok{log.cars, pollution}\SpecialCharTok{$}\NormalTok{log.no2)}
\end{Highlighting}
\end{Shaded}

\includegraphics{mandatory_assignment1_files/figure-latex/unnamed-chunk-9-1.pdf}

\begin{Shaded}
\begin{Highlighting}[]
\FunctionTok{print}\NormalTok{(}\StringTok{"Correlation:"}\NormalTok{)}
\end{Highlighting}
\end{Shaded}

\begin{verbatim}
## [1] "Correlation:"
\end{verbatim}

\begin{Shaded}
\begin{Highlighting}[]
\FunctionTok{cor}\NormalTok{(pollution}\SpecialCharTok{$}\NormalTok{log.cars, pollution}\SpecialCharTok{$}\NormalTok{log.no2)}
\end{Highlighting}
\end{Shaded}

\begin{verbatim}
## [1] 0.5120504
\end{verbatim}

\hypertarget{discussion}{%
\subsubsection{Discussion:}\label{discussion}}

\hypertarget{b}{%
\subsection{b)}\label{b}}

\hypertarget{method-1}{%
\subsubsection{Method:}\label{method-1}}

A linear model is fitted, with the NO2 levels as dependent and cars/hour
as dependent variable:

\begin{Shaded}
\begin{Highlighting}[]
\NormalTok{pollution.fit }\OtherTok{=} \FunctionTok{lm}\NormalTok{(log.no2}\SpecialCharTok{\textasciitilde{}}\NormalTok{log.cars, }\AttributeTok{data=}\NormalTok{pollution)}
\FunctionTok{summary}\NormalTok{(pollution.fit)}
\end{Highlighting}
\end{Shaded}

\begin{verbatim}
## 
## Call:
## lm(formula = log.no2 ~ log.cars, data = pollution)
## 
## Residuals:
##      Min       1Q   Median       3Q      Max 
## -2.18822 -0.40071  0.06428  0.40362  2.48472 
## 
## Coefficients:
##             Estimate Std. Error t value Pr(>|t|)    
## (Intercept)  1.23310    0.18755   6.575 1.23e-10 ***
## log.cars     0.35353    0.02657  13.303  < 2e-16 ***
## ---
## Signif. codes:  0 '***' 0.001 '**' 0.01 '*' 0.05 '.' 0.1 ' ' 1
## 
## Residual standard error: 0.6454 on 498 degrees of freedom
## Multiple R-squared:  0.2622, Adjusted R-squared:  0.2607 
## F-statistic:   177 on 1 and 498 DF,  p-value: < 2.2e-16
\end{verbatim}

The observations are plotted together with a line representing the
fitted linear regression model:

\begin{Shaded}
\begin{Highlighting}[]
\FunctionTok{plot}\NormalTok{(pollution}\SpecialCharTok{$}\NormalTok{log.cars, pollution}\SpecialCharTok{$}\NormalTok{log.no2)}
\FunctionTok{abline}\NormalTok{(pollution.fit)}
\end{Highlighting}
\end{Shaded}

\includegraphics{mandatory_assignment1_files/figure-latex/unnamed-chunk-11-1.pdf}
\#\#\# Discussion) R\^{}2, the coefficient of determination, tells us
how much of the of the variance present in the dependent variable can be
explained by the model/the independent variables. A high R\^{}2 value
indicates that a high amount of the dependent variable's variance can be
explained by the model, and therefore a good representation.

\hypertarget{c}{%
\subsection{c)}\label{c}}

For a linear regression we assume: * linearity between the dependent and
independent variable * Homoscedacity - constant variance in the
residuals * Normally distributed residuals * Uncorrelated errors (This
assumptions I have not testes for, as this not has been part of the
curriculum yet)

\hypertarget{method-2}{%
\subsubsection{Method:}\label{method-2}}

For the first assumption, we plot a CPR (component-plus-residual) plot:

\begin{Shaded}
\begin{Highlighting}[]
\FunctionTok{crPlots}\NormalTok{(pollution.fit)}
\end{Highlighting}
\end{Shaded}

\includegraphics{mandatory_assignment1_files/figure-latex/unnamed-chunk-12-1.pdf}
For the homoscedacity check, the residuals vs fitted values are plotted.
If there is no systematic pattern in the residuals, the assumption is
fulfilled.

\begin{Shaded}
\begin{Highlighting}[]
\FunctionTok{plot}\NormalTok{(pollution.fit, }\DecValTok{1}\NormalTok{)}
\end{Highlighting}
\end{Shaded}

\includegraphics{mandatory_assignment1_files/figure-latex/unnamed-chunk-13-1.pdf}

\begin{Shaded}
\begin{Highlighting}[]
\FunctionTok{plot}\NormalTok{(pollution.fit, }\DecValTok{3}\NormalTok{)}
\end{Highlighting}
\end{Shaded}

\includegraphics{mandatory_assignment1_files/figure-latex/unnamed-chunk-13-2.pdf}
For the linearity assumption, we can use histograms, boxplots and Q-Q
plots:

\begin{Shaded}
\begin{Highlighting}[]
\FunctionTok{plotNormalHistogram}\NormalTok{(pollution.fit}\SpecialCharTok{$}\NormalTok{residuals, }\AttributeTok{main=}\StringTok{"Histogram of residuals with normal distribution"}\NormalTok{)}
\end{Highlighting}
\end{Shaded}

\includegraphics{mandatory_assignment1_files/figure-latex/unnamed-chunk-14-1.pdf}

\begin{Shaded}
\begin{Highlighting}[]
\FunctionTok{boxplot}\NormalTok{(pollution.fit}\SpecialCharTok{$}\NormalTok{residuals)}
\end{Highlighting}
\end{Shaded}

\includegraphics{mandatory_assignment1_files/figure-latex/unnamed-chunk-14-2.pdf}

\begin{Shaded}
\begin{Highlighting}[]
\FunctionTok{plot}\NormalTok{(pollution.fit, }\DecValTok{2}\NormalTok{)}
\end{Highlighting}
\end{Shaded}

\includegraphics{mandatory_assignment1_files/figure-latex/unnamed-chunk-14-3.pdf}
\#\#\# Discussion:

\hypertarget{d}{%
\subsection{d)}\label{d}}

\hypertarget{method-3}{%
\subsubsection{Method:}\label{method-3}}

Im applying the backwards-method in order to exclude predictors in
possible multiple models. I start by including all variables and exclude
one and one based on their significance/p-value. The models are being
evaluated based on the R\^{}2. I test both applying a log-transformation
to the variables, as well as including their the second degree terms.

Model 1:

\begin{Shaded}
\begin{Highlighting}[]
\NormalTok{pollution.fit}\FloatTok{.1}\OtherTok{=}\FunctionTok{lm}\NormalTok{(log.no2}\SpecialCharTok{\textasciitilde{}}\NormalTok{log.cars}\SpecialCharTok{+}\NormalTok{temp}\SpecialCharTok{+}\NormalTok{wind.speed}\SpecialCharTok{+}\NormalTok{hour.of.day, }\AttributeTok{data=}\NormalTok{pollution)}
\FunctionTok{summary}\NormalTok{(pollution.fit}\FloatTok{.1}\NormalTok{)}
\end{Highlighting}
\end{Shaded}

\begin{verbatim}
## 
## Call:
## lm(formula = log.no2 ~ log.cars + temp + wind.speed + hour.of.day, 
##     data = pollution)
## 
## Residuals:
##      Min       1Q   Median       3Q      Max 
## -2.24876 -0.32070  0.03084  0.33860  1.96057 
## 
## Coefficients:
##              Estimate Std. Error t value Pr(>|t|)    
## (Intercept)  1.152131   0.175045   6.582 1.19e-10 ***
## log.cars     0.456974   0.028411  16.084  < 2e-16 ***
## temp        -0.026855   0.003905  -6.877 1.85e-11 ***
## wind.speed  -0.149334   0.014076 -10.609  < 2e-16 ***
## hour.of.day -0.013025   0.004452  -2.926   0.0036 ** 
## ---
## Signif. codes:  0 '***' 0.001 '**' 0.01 '*' 0.05 '.' 0.1 ' ' 1
## 
## Residual standard error: 0.5508 on 495 degrees of freedom
## Multiple R-squared:  0.4658, Adjusted R-squared:  0.4615 
## F-statistic: 107.9 on 4 and 495 DF,  p-value: < 2.2e-16
\end{verbatim}

Model 3:

\begin{Shaded}
\begin{Highlighting}[]
\NormalTok{pollution.fit}\FloatTok{.3}\OtherTok{=}\FunctionTok{lm}\NormalTok{(log.no2}\SpecialCharTok{\textasciitilde{}}\NormalTok{log.cars}\SpecialCharTok{+}\NormalTok{temp}\SpecialCharTok{+}\NormalTok{wind.speed, }\AttributeTok{data=}\NormalTok{pollution)}
\FunctionTok{summary}\NormalTok{(pollution.fit}\FloatTok{.3}\NormalTok{)}
\end{Highlighting}
\end{Shaded}

\begin{verbatim}
## 
## Call:
## lm(formula = log.no2 ~ log.cars + temp + wind.speed, data = pollution)
## 
## Residuals:
##      Min       1Q   Median       3Q      Max 
## -2.13980 -0.33142  0.04882  0.35257  1.97666 
## 
## Coefficients:
##              Estimate Std. Error t value Pr(>|t|)    
## (Intercept)  1.316491   0.167043   7.881 2.07e-14 ***
## log.cars     0.409026   0.023384  17.492  < 2e-16 ***
## temp        -0.026447   0.003932  -6.725 4.83e-11 ***
## wind.speed  -0.146594   0.014152 -10.359  < 2e-16 ***
## ---
## Signif. codes:  0 '***' 0.001 '**' 0.01 '*' 0.05 '.' 0.1 ' ' 1
## 
## Residual standard error: 0.555 on 496 degrees of freedom
## Multiple R-squared:  0.4566, Adjusted R-squared:  0.4533 
## F-statistic: 138.9 on 3 and 496 DF,  p-value: < 2.2e-16
\end{verbatim}

Model 4:

\begin{Shaded}
\begin{Highlighting}[]
\NormalTok{pollution.fit}\FloatTok{.4}\OtherTok{=}\FunctionTok{lm}\NormalTok{(log.no2}\SpecialCharTok{\textasciitilde{}}\NormalTok{log.cars}\SpecialCharTok{+}\NormalTok{wind.speed, }\AttributeTok{data=}\NormalTok{pollution)}
\FunctionTok{summary}\NormalTok{(pollution.fit}\FloatTok{.4}\NormalTok{)}
\end{Highlighting}
\end{Shaded}

\begin{verbatim}
## 
## Call:
## lm(formula = log.no2 ~ log.cars + wind.speed, data = pollution)
## 
## Residuals:
##      Min       1Q   Median       3Q      Max 
## -2.03158 -0.34600  0.01792  0.39106  2.10633 
## 
## Coefficients:
##             Estimate Std. Error t value Pr(>|t|)    
## (Intercept)  1.54518    0.17067   9.054   <2e-16 ***
## log.cars     0.37928    0.02396  15.828   <2e-16 ***
## wind.speed  -0.16088    0.01460 -11.019   <2e-16 ***
## ---
## Signif. codes:  0 '***' 0.001 '**' 0.01 '*' 0.05 '.' 0.1 ' ' 1
## 
## Residual standard error: 0.5791 on 497 degrees of freedom
## Multiple R-squared:  0.407,  Adjusted R-squared:  0.4047 
## F-statistic: 170.6 on 2 and 497 DF,  p-value: < 2.2e-16
\end{verbatim}

In order to see if a log transformation might affect the relationship
between the dependent and independent variables, the not transformed
variables are also transformed. Temperature is not transformed, as it
has negative values.

Model 5:

\begin{Shaded}
\begin{Highlighting}[]
\NormalTok{pollution.fit}\FloatTok{.5}\OtherTok{=}\FunctionTok{lm}\NormalTok{(log.no2}\SpecialCharTok{\textasciitilde{}}\NormalTok{log.cars}\SpecialCharTok{+}\NormalTok{temp}\SpecialCharTok{+}\FunctionTok{log}\NormalTok{(wind.speed)}\SpecialCharTok{+}\FunctionTok{log}\NormalTok{(hour.of.day), }\AttributeTok{data=}\NormalTok{pollution)}
\FunctionTok{summary}\NormalTok{(pollution.fit}\FloatTok{.5}\NormalTok{)}
\end{Highlighting}
\end{Shaded}

\begin{verbatim}
## 
## Call:
## lm(formula = log.no2 ~ log.cars + temp + log(wind.speed) + log(hour.of.day), 
##     data = pollution)
## 
## Residuals:
##      Min       1Q   Median       3Q      Max 
## -2.26406 -0.31784  0.04366  0.34532  1.83379 
## 
## Coefficients:
##                   Estimate Std. Error t value Pr(>|t|)    
## (Intercept)       1.111953   0.169410   6.564 1.33e-10 ***
## log.cars          0.461040   0.031041  14.852  < 2e-16 ***
## temp             -0.026922   0.003853  -6.988 9.07e-12 ***
## log(wind.speed)  -0.415333   0.036410 -11.407  < 2e-16 ***
## log(hour.of.day) -0.098007   0.041880  -2.340   0.0197 *  
## ---
## Signif. codes:  0 '***' 0.001 '**' 0.01 '*' 0.05 '.' 0.1 ' ' 1
## 
## Residual standard error: 0.5431 on 495 degrees of freedom
## Multiple R-squared:  0.4807, Adjusted R-squared:  0.4766 
## F-statistic: 114.6 on 4 and 495 DF,  p-value: < 2.2e-16
\end{verbatim}

Model 6:

\begin{Shaded}
\begin{Highlighting}[]
\NormalTok{pollution.fit}\FloatTok{.6}\OtherTok{=}\FunctionTok{lm}\NormalTok{(log.no2}\SpecialCharTok{\textasciitilde{}}\NormalTok{log.cars}\SpecialCharTok{+}\NormalTok{temp}\SpecialCharTok{+}\FunctionTok{log}\NormalTok{(wind.speed), }\AttributeTok{data=}\NormalTok{pollution)}
\FunctionTok{summary}\NormalTok{(pollution.fit}\FloatTok{.6}\NormalTok{)}
\end{Highlighting}
\end{Shaded}

\begin{verbatim}
## 
## Call:
## lm(formula = log.no2 ~ log.cars + temp + log(wind.speed), data = pollution)
## 
## Residuals:
##      Min       1Q   Median       3Q      Max 
## -2.07759 -0.33892  0.05458  0.36666  1.83266 
## 
## Coefficients:
##                  Estimate Std. Error t value Pr(>|t|)    
## (Intercept)      1.229009   0.162586   7.559 1.98e-13 ***
## log.cars         0.411979   0.022995  17.916  < 2e-16 ***
## temp            -0.026304   0.003861  -6.813 2.79e-11 ***
## log(wind.speed) -0.414496   0.036572 -11.334  < 2e-16 ***
## ---
## Signif. codes:  0 '***' 0.001 '**' 0.01 '*' 0.05 '.' 0.1 ' ' 1
## 
## Residual standard error: 0.5455 on 496 degrees of freedom
## Multiple R-squared:  0.475,  Adjusted R-squared:  0.4718 
## F-statistic: 149.6 on 3 and 496 DF,  p-value: < 2.2e-16
\end{verbatim}

Testing if including the second degree term of the variables will
improve the model. Model 7:

\begin{Shaded}
\begin{Highlighting}[]
\NormalTok{pollution.fit}\FloatTok{.7}\OtherTok{=}\FunctionTok{lm}\NormalTok{(log.no2}\SpecialCharTok{\textasciitilde{}}\NormalTok{log.cars}\SpecialCharTok{+}\FunctionTok{I}\NormalTok{(log.cars}\SpecialCharTok{\^{}}\DecValTok{2}\NormalTok{)}\SpecialCharTok{+}\NormalTok{temp}\SpecialCharTok{+}\FunctionTok{I}\NormalTok{(temp}\SpecialCharTok{\^{}}\DecValTok{2}\NormalTok{)}\SpecialCharTok{+}\NormalTok{wind.speed}\SpecialCharTok{+}\FunctionTok{I}\NormalTok{(wind.speed}\SpecialCharTok{\^{}}\DecValTok{2}\NormalTok{)}\SpecialCharTok{+}\NormalTok{hour.of.day}\SpecialCharTok{+}\FunctionTok{I}\NormalTok{(hour.of.day}\SpecialCharTok{\^{}}\DecValTok{2}\NormalTok{), }\AttributeTok{data=}\NormalTok{pollution)}
\FunctionTok{summary}\NormalTok{(pollution.fit}\FloatTok{.7}\NormalTok{)}
\end{Highlighting}
\end{Shaded}

\begin{verbatim}
## 
## Call:
## lm(formula = log.no2 ~ log.cars + I(log.cars^2) + temp + I(temp^2) + 
##     wind.speed + I(wind.speed^2) + hour.of.day + I(hour.of.day^2), 
##     data = pollution)
## 
## Residuals:
##      Min       1Q   Median       3Q      Max 
## -2.07791 -0.32530  0.02189  0.36192  1.85442 
## 
## Coefficients:
##                    Estimate Std. Error t value Pr(>|t|)    
## (Intercept)       4.7410924  1.0699553   4.431 1.16e-05 ***
## log.cars         -0.5744937  0.3322757  -1.729  0.08444 .  
## I(log.cars^2)     0.0799072  0.0253074   3.157  0.00169 ** 
## temp             -0.0281999  0.0039124  -7.208 2.16e-12 ***
## I(temp^2)         0.0002639  0.0003775   0.699  0.48480    
## wind.speed       -0.3749046  0.0436121  -8.596  < 2e-16 ***
## I(wind.speed^2)   0.0297334  0.0054763   5.430 8.90e-08 ***
## hour.of.day      -0.0329595  0.0244929  -1.346  0.17903    
## I(hour.of.day^2)  0.0008904  0.0008744   1.018  0.30899    
## ---
## Signif. codes:  0 '***' 0.001 '**' 0.01 '*' 0.05 '.' 0.1 ' ' 1
## 
## Residual standard error: 0.5331 on 491 degrees of freedom
## Multiple R-squared:  0.5036, Adjusted R-squared:  0.4955 
## F-statistic: 62.26 on 8 and 491 DF,  p-value: < 2.2e-16
\end{verbatim}

Model 8:

\begin{Shaded}
\begin{Highlighting}[]
\NormalTok{pollution.fit}\FloatTok{.8}\OtherTok{=}\FunctionTok{lm}\NormalTok{(log.no2}\SpecialCharTok{\textasciitilde{}}\NormalTok{log.cars}\SpecialCharTok{+}\FunctionTok{I}\NormalTok{(log.cars}\SpecialCharTok{\^{}}\DecValTok{2}\NormalTok{)}\SpecialCharTok{+}\NormalTok{temp}\SpecialCharTok{+}\FunctionTok{I}\NormalTok{(temp}\SpecialCharTok{\^{}}\DecValTok{2}\NormalTok{)}\SpecialCharTok{+}\NormalTok{wind.speed}\SpecialCharTok{+}\FunctionTok{I}\NormalTok{(wind.speed}\SpecialCharTok{\^{}}\DecValTok{2}\NormalTok{)}\SpecialCharTok{+}\NormalTok{hour.of.day, }\AttributeTok{data=}\NormalTok{pollution)}
\FunctionTok{summary}\NormalTok{(pollution.fit}\FloatTok{.8}\NormalTok{)}
\end{Highlighting}
\end{Shaded}

\begin{verbatim}
## 
## Call:
## lm(formula = log.no2 ~ log.cars + I(log.cars^2) + temp + I(temp^2) + 
##     wind.speed + I(wind.speed^2) + hour.of.day, data = pollution)
## 
## Residuals:
##      Min       1Q   Median       3Q      Max 
## -2.00476 -0.33674  0.02924  0.35946  1.82597 
## 
## Coefficients:
##                   Estimate Std. Error t value Pr(>|t|)    
## (Intercept)      4.6066558  1.0618207   4.338 1.74e-05 ***
## log.cars        -0.5365889  0.3301970  -1.625  0.10479    
## I(log.cars^2)    0.0748395  0.0248142   3.016  0.00269 ** 
## temp            -0.0278030  0.0038931  -7.142 3.34e-12 ***
## I(temp^2)        0.0002075  0.0003734   0.556  0.57869    
## wind.speed      -0.3734395  0.0435900  -8.567  < 2e-16 ***
## I(wind.speed^2)  0.0292781  0.0054582   5.364 1.25e-07 ***
## hour.of.day     -0.0084495  0.0045468  -1.858  0.06372 .  
## ---
## Signif. codes:  0 '***' 0.001 '**' 0.01 '*' 0.05 '.' 0.1 ' ' 1
## 
## Residual standard error: 0.5332 on 492 degrees of freedom
## Multiple R-squared:  0.5025, Adjusted R-squared:  0.4954 
## F-statistic:    71 on 7 and 492 DF,  p-value: < 2.2e-16
\end{verbatim}

Model9:

\begin{Shaded}
\begin{Highlighting}[]
\NormalTok{pollution.fit}\FloatTok{.9}\OtherTok{=}\FunctionTok{lm}\NormalTok{(log.no2}\SpecialCharTok{\textasciitilde{}}\NormalTok{log.cars}\SpecialCharTok{+}\FunctionTok{I}\NormalTok{(log.cars}\SpecialCharTok{\^{}}\DecValTok{2}\NormalTok{)}\SpecialCharTok{+}\NormalTok{temp}\SpecialCharTok{+}\NormalTok{wind.speed}\SpecialCharTok{+}\FunctionTok{I}\NormalTok{(wind.speed}\SpecialCharTok{\^{}}\DecValTok{2}\NormalTok{)}\SpecialCharTok{+}\NormalTok{hour.of.day, }\AttributeTok{data=}\NormalTok{pollution)}
\FunctionTok{summary}\NormalTok{(pollution.fit}\FloatTok{.9}\NormalTok{)}
\end{Highlighting}
\end{Shaded}

\begin{verbatim}
## 
## Call:
## lm(formula = log.no2 ~ log.cars + I(log.cars^2) + temp + wind.speed + 
##     I(wind.speed^2) + hour.of.day, data = pollution)
## 
## Residuals:
##      Min       1Q   Median       3Q      Max 
## -2.00854 -0.33453  0.02454  0.35820  1.82492 
## 
## Coefficients:
##                  Estimate Std. Error t value Pr(>|t|)    
## (Intercept)      4.620939   1.060765   4.356 1.61e-05 ***
## log.cars        -0.539959   0.329910  -1.637  0.10233    
## I(log.cars^2)    0.075157   0.024790   3.032  0.00256 ** 
## temp            -0.027290   0.003779  -7.221 1.97e-12 ***
## wind.speed      -0.371854   0.043466  -8.555  < 2e-16 ***
## I(wind.speed^2)  0.029067   0.005441   5.342 1.41e-07 ***
## hour.of.day     -0.008471   0.004543  -1.864  0.06286 .  
## ---
## Signif. codes:  0 '***' 0.001 '**' 0.01 '*' 0.05 '.' 0.1 ' ' 1
## 
## Residual standard error: 0.5328 on 493 degrees of freedom
## Multiple R-squared:  0.5022, Adjusted R-squared:  0.4961 
## F-statistic: 82.89 on 6 and 493 DF,  p-value: < 2.2e-16
\end{verbatim}

Model 10:

\begin{Shaded}
\begin{Highlighting}[]
\NormalTok{pollution.fit}\FloatTok{.10}\OtherTok{=}\FunctionTok{lm}\NormalTok{(log.no2}\SpecialCharTok{\textasciitilde{}}\FunctionTok{I}\NormalTok{(log.cars}\SpecialCharTok{\^{}}\DecValTok{2}\NormalTok{)}\SpecialCharTok{+}\NormalTok{temp}\SpecialCharTok{+}\NormalTok{wind.speed}\SpecialCharTok{+}\FunctionTok{I}\NormalTok{(wind.speed}\SpecialCharTok{\^{}}\DecValTok{2}\NormalTok{)}\SpecialCharTok{+}\NormalTok{hour.of.day, }\AttributeTok{data=}\NormalTok{pollution)}
\FunctionTok{summary}\NormalTok{(pollution.fit}\FloatTok{.10}\NormalTok{)}
\end{Highlighting}
\end{Shaded}

\begin{verbatim}
## 
## Call:
## lm(formula = log.no2 ~ I(log.cars^2) + temp + wind.speed + I(wind.speed^2) + 
##     hour.of.day, data = pollution)
## 
## Residuals:
##      Min       1Q   Median       3Q      Max 
## -2.08213 -0.32288  0.02327  0.34721  1.81603 
## 
## Coefficients:
##                  Estimate Std. Error t value Pr(>|t|)    
## (Intercept)      2.894267   0.110817  26.117  < 2e-16 ***
## I(log.cars^2)    0.034724   0.002069  16.786  < 2e-16 ***
## temp            -0.027190   0.003785  -7.183 2.52e-12 ***
## wind.speed      -0.367001   0.043438  -8.449 3.32e-16 ***
## I(wind.speed^2)  0.028565   0.005442   5.249 2.27e-07 ***
## hour.of.day     -0.011171   0.004240  -2.634  0.00869 ** 
## ---
## Signif. codes:  0 '***' 0.001 '**' 0.01 '*' 0.05 '.' 0.1 ' ' 1
## 
## Residual standard error: 0.5337 on 494 degrees of freedom
## Multiple R-squared:  0.4995, Adjusted R-squared:  0.4944 
## F-statistic:  98.6 on 5 and 494 DF,  p-value: < 2.2e-16
\end{verbatim}

MOdel 11:

\begin{Shaded}
\begin{Highlighting}[]
\NormalTok{pollution.fit}\FloatTok{.11}\OtherTok{=}\FunctionTok{lm}\NormalTok{(log.no2}\SpecialCharTok{\textasciitilde{}}\FunctionTok{I}\NormalTok{(log.cars}\SpecialCharTok{\^{}}\DecValTok{2}\NormalTok{)}\SpecialCharTok{+}\NormalTok{temp}\SpecialCharTok{+}\NormalTok{wind.speed}\SpecialCharTok{+}\FunctionTok{I}\NormalTok{(wind.speed}\SpecialCharTok{\^{}}\DecValTok{2}\NormalTok{), }\AttributeTok{data=}\NormalTok{pollution)}
\FunctionTok{summary}\NormalTok{(pollution.fit}\FloatTok{.11}\NormalTok{)}
\end{Highlighting}
\end{Shaded}

\begin{verbatim}
## 
## Call:
## lm(formula = log.no2 ~ I(log.cars^2) + temp + wind.speed + I(wind.speed^2), 
##     data = pollution)
## 
## Residuals:
##      Min       1Q   Median       3Q      Max 
## -1.99114 -0.33111  0.04251  0.36930  1.82765 
## 
## Coefficients:
##                  Estimate Std. Error t value Pr(>|t|)    
## (Intercept)      2.900016   0.111459  26.019  < 2e-16 ***
## I(log.cars^2)    0.031692   0.001729  18.327  < 2e-16 ***
## temp            -0.026870   0.003806  -7.060 5.66e-12 ***
## wind.speed      -0.365108   0.043692  -8.356 6.57e-16 ***
## I(wind.speed^2)  0.028634   0.005474   5.231 2.50e-07 ***
## ---
## Signif. codes:  0 '***' 0.001 '**' 0.01 '*' 0.05 '.' 0.1 ' ' 1
## 
## Residual standard error: 0.5369 on 495 degrees of freedom
## Multiple R-squared:  0.4925, Adjusted R-squared:  0.4884 
## F-statistic: 120.1 on 4 and 495 DF,  p-value: < 2.2e-16
\end{verbatim}

\hypertarget{discussion-1}{%
\subsubsection{Discussion:}\label{discussion-1}}

Based on the models, log-transforming the variables will slightly
improve the R\^{}2 of the best non-transformed model (model 1: 0.4658,
model 5: 0.4807). Both models included all variables, but we can see
that the model's R\^{}2 does not decrease a lot when removing hour of
day. Model 7 includes the second degree term of all variables, and
results in a R\^{}2=0.5036. But in model 11 we can see that we can
remove some of the variables in model 7 and still achieve a relative
high R\^{}2 value, 0.4925. I choose to continue with model 11, as this
model is also less complex/includes less variables than example model 7.

\hypertarget{e}{%
\subsection{e)}\label{e}}

\hypertarget{method-4}{%
\subsubsection{Method:}\label{method-4}}

In order to check the assumptions for model 11, I use the same methods
as in Excersice 1c)-plotting CPR plots for the independent variables to
check for linearity, plotting the residuals vs fitted values to check
for constant variance and plotting the histogram, boxplot and qq-plot of
the residuals to examine whether they are normally distributed:

The CPR plots. I have also included model 1's CPR plots to compare the
effect of including the second degree terms:

\begin{Shaded}
\begin{Highlighting}[]
\FunctionTok{crPlots}\NormalTok{(pollution.fit}\FloatTok{.11}\NormalTok{)}
\end{Highlighting}
\end{Shaded}

\includegraphics{mandatory_assignment1_files/figure-latex/unnamed-chunk-25-1.pdf}

\begin{Shaded}
\begin{Highlighting}[]
\FunctionTok{crPlots}\NormalTok{(pollution.fit}\FloatTok{.1}\NormalTok{)}
\end{Highlighting}
\end{Shaded}

\includegraphics{mandatory_assignment1_files/figure-latex/unnamed-chunk-25-2.pdf}
The residual vs.~fitted plots:

\begin{Shaded}
\begin{Highlighting}[]
\FunctionTok{plot}\NormalTok{(pollution.fit}\FloatTok{.11}\NormalTok{, }\DecValTok{1}\NormalTok{)}
\end{Highlighting}
\end{Shaded}

\includegraphics{mandatory_assignment1_files/figure-latex/unnamed-chunk-26-1.pdf}

\begin{Shaded}
\begin{Highlighting}[]
\FunctionTok{plot}\NormalTok{(pollution.fit}\FloatTok{.11}\NormalTok{, }\DecValTok{3}\NormalTok{)}
\end{Highlighting}
\end{Shaded}

\includegraphics{mandatory_assignment1_files/figure-latex/unnamed-chunk-26-2.pdf}
The histogram, boxplot and qq-plot of the residuals:

\begin{Shaded}
\begin{Highlighting}[]
\FunctionTok{plotNormalHistogram}\NormalTok{(pollution.fit}\FloatTok{.11}\SpecialCharTok{$}\NormalTok{residuals, }\AttributeTok{main=}\StringTok{"Histogram of residuals with normal distribution"}\NormalTok{)}
\end{Highlighting}
\end{Shaded}

\includegraphics{mandatory_assignment1_files/figure-latex/unnamed-chunk-27-1.pdf}

\begin{Shaded}
\begin{Highlighting}[]
\FunctionTok{boxplot}\NormalTok{(pollution.fit}\FloatTok{.11}\SpecialCharTok{$}\NormalTok{residuals)}
\end{Highlighting}
\end{Shaded}

\includegraphics{mandatory_assignment1_files/figure-latex/unnamed-chunk-27-2.pdf}

\begin{Shaded}
\begin{Highlighting}[]
\FunctionTok{plot}\NormalTok{(pollution.fit}\FloatTok{.11}\NormalTok{, }\DecValTok{2}\NormalTok{)}
\end{Highlighting}
\end{Shaded}

\includegraphics{mandatory_assignment1_files/figure-latex/unnamed-chunk-27-3.pdf}

\hypertarget{discussion-2}{%
\subsubsection{Discussion:}\label{discussion-2}}

Interpreting the model:

Comment on the assumption checks:

\hypertarget{excersice-2}{%
\section{Excersice 2:}\label{excersice-2}}

Reading the blood-dataset, and defining the age-variable as categorical:

\begin{Shaded}
\begin{Highlighting}[]
\NormalTok{blood}\OtherTok{=}\FunctionTok{read.table}\NormalTok{(}\StringTok{"https://www.uio.no/studier/emner/matnat/math/STK4900/data/blood.txt"}\NormalTok{, }\AttributeTok{header=}\ConstantTok{TRUE}\NormalTok{, }\AttributeTok{colClasses =} \FunctionTok{c}\NormalTok{(}\StringTok{"numeric"}\NormalTok{, }\StringTok{"factor"}\NormalTok{))}
\end{Highlighting}
\end{Shaded}

\hypertarget{a-1}{%
\subsection{a)}\label{a-1}}

\hypertarget{method-5}{%
\subsubsection{Method:}\label{method-5}}

I first inspect the mean, sd, median, IQR, min, max, 1.Qr and 3. Qr per
age group. Also a histogram and boxplot of the groups bloodpressure are
printed.

\begin{Shaded}
\begin{Highlighting}[]
\FunctionTok{setDT}\NormalTok{(blood)}
\NormalTok{blood[, }\FunctionTok{as.list}\NormalTok{(}\FunctionTok{summary}\NormalTok{(Bloodpr)), by }\OtherTok{=}\NormalTok{ age]}
\end{Highlighting}
\end{Shaded}

\begin{verbatim}
##    age Min. 1st Qu. Median     Mean 3rd Qu. Max.
## 1:   1  104   112.0    117 122.1667  129.00  160
## 2:   2  108   121.5    137 139.0833  157.75  174
## 3:   3  110   138.0    148 155.1667  164.00  214
\end{verbatim}

\begin{Shaded}
\begin{Highlighting}[]
\NormalTok{blood[ ,}\FunctionTok{list}\NormalTok{(}\AttributeTok{sd=}\FunctionTok{sd}\NormalTok{(Bloodpr)), by}\OtherTok{=}\NormalTok{age]}
\end{Highlighting}
\end{Shaded}

\begin{verbatim}
##    age       sd
## 1:   1 15.33761
## 2:   2 22.62524
## 3:   3 27.71883
\end{verbatim}

\begin{Shaded}
\begin{Highlighting}[]
\NormalTok{blood[ ,}\FunctionTok{list}\NormalTok{(}\AttributeTok{IQR=}\FunctionTok{IQR}\NormalTok{(Bloodpr)), by}\OtherTok{=}\NormalTok{age]}
\end{Highlighting}
\end{Shaded}

\begin{verbatim}
##    age   IQR
## 1:   1 17.00
## 2:   2 36.25
## 3:   3 26.00
\end{verbatim}

Plots by groups:

\begin{Shaded}
\begin{Highlighting}[]
\FunctionTok{ggplot}\NormalTok{(blood, }\FunctionTok{aes}\NormalTok{(}\AttributeTok{x=}\NormalTok{age, }\AttributeTok{y=}\NormalTok{Bloodpr)) }\SpecialCharTok{+} 
    \FunctionTok{geom\_boxplot}\NormalTok{()}
\end{Highlighting}
\end{Shaded}

\includegraphics{mandatory_assignment1_files/figure-latex/unnamed-chunk-30-1.pdf}

\begin{Shaded}
\begin{Highlighting}[]
\FunctionTok{ggplot}\NormalTok{(blood, }\FunctionTok{aes}\NormalTok{(}\AttributeTok{x =}\NormalTok{ Bloodpr)) }\SpecialCharTok{+}
  \FunctionTok{geom\_histogram}\NormalTok{(}\AttributeTok{fill =} \StringTok{"white"}\NormalTok{, }\AttributeTok{colour =} \StringTok{"black"}\NormalTok{) }\SpecialCharTok{+}
  \FunctionTok{facet\_grid}\NormalTok{(age }\SpecialCharTok{\textasciitilde{}}\NormalTok{ .)}
\end{Highlighting}
\end{Shaded}

\begin{verbatim}
## `stat_bin()` using `bins = 30`. Pick better value with `binwidth`.
\end{verbatim}

\includegraphics{mandatory_assignment1_files/figure-latex/unnamed-chunk-30-2.pdf}

\hypertarget{discussion-3}{%
\subsubsection{Discussion:}\label{discussion-3}}

\hypertarget{b-1}{%
\subsection{b)}\label{b-1}}

\hypertarget{method-6}{%
\subsubsection{Method:}\label{method-6}}

A two-way ANOVA is performed to examine whether there is a difference in
the expected blood pressure across the groups:

\begin{Shaded}
\begin{Highlighting}[]
\NormalTok{aov.blood }\OtherTok{=} \FunctionTok{aov}\NormalTok{(Bloodpr}\SpecialCharTok{\textasciitilde{}}\NormalTok{age, }\AttributeTok{data=}\NormalTok{blood)}
\FunctionTok{summary}\NormalTok{(aov.blood)}
\end{Highlighting}
\end{Shaded}

\begin{verbatim}
##             Df Sum Sq Mean Sq F value  Pr(>F)   
## age          2   6535    3268   6.469 0.00426 **
## Residuals   33  16670     505                   
## ---
## Signif. codes:  0 '***' 0.001 '**' 0.01 '*' 0.05 '.' 0.1 ' ' 1
\end{verbatim}

\hypertarget{discussion-4}{%
\subsubsection{Discussion:}\label{discussion-4}}

The assumptions that are done for a two-way ANOVA, are:\textbackslash{}
- The observations are independent of each other. As the samples are
random, and one person cannot be in multiple age groups, I assume that
the assumption is fulfilled. \textbackslash{} - The dependent variable
should be near-to-normally distributed for each group. This is hard to
controll with such a small sample size. The histogram of the groups can
be examined, but will not give a conclusive answer. From the histogram
in 2a), one can see that the observations in the groups is very spread,
only group 1 looks like something that could resemble a normal
distribution. I am therefore nut fully sure if the assumption is
fulfilled for this specific sample of people, and believe one would need
more samples to determine this. But the test is robust to some deviation
from the normal distribution, so I continue the analysis of the ANOVA.
\textbackslash{} - That the variance for the groups are similar. One can
test for this assumption by using for example levenes test. But it needs
to be mentioned that the sample size of the dataset is relative small
(12 obs. in each group), and it can therefore be expected that the
group's variance is unequal. We instead consider the groups' samplesize.
As these are the same, I may assume the assumption as fulfilled.
\textbackslash{}

In the hypothesis test, the null hypothesis states that the average
expected blood pressure of all groups is the same. The alternative test
states that one or more groups might have a significantly different
expected average blood pressure compared to the other groups. A high
F-value for 2 and 33 degrees of freedom might indicate that we can
reject the null hypothesis, and that one or more groups' expected
average outcome differs from the others. According to the performed
ANOVA test, a low p-value is returned (p=0.00426) and we can reject the
null hypothesis with a 99\% significance level. In order to evaluate
which of the groups differs from the others, we need to perform a
regression analysis and evaluate the expected outcome. This is performed
in 2c). As mentioned earlier, based on the boxplot of the groups, it can
be expected that at least group 1 and group 3 will differ from each
other, but I am less confident whether group 2's expected blood pressure
will be significantly different relative to the other groups.
\textbackslash{}

\hypertarget{c-1}{%
\subsection{c)}\label{c-1}}

\hypertarget{method-7}{%
\subsubsection{Method:}\label{method-7}}

In order to compare the expected blood pressure for the groups a
regression model is fit, using the blood pressure as dependent variable
and age as independent variable. A treatment-contrast is used with age
group 1 as reference, meaning the intercept returns the expected outcome
in the reference group, while for group 2 and 3 the expected outcome
relative to group 1 is returned (expected blood pressure in group 2 -
expected blood pressure in group 1,and expected blood pressure in group
3 - expected blood pressure in group 1).

\begin{Shaded}
\begin{Highlighting}[]
\NormalTok{blood.fit }\OtherTok{=} \FunctionTok{lm}\NormalTok{(Bloodpr}\SpecialCharTok{\textasciitilde{}}\NormalTok{age, }\AttributeTok{data =}\NormalTok{ blood)}
\FunctionTok{summary}\NormalTok{(blood.fit)}
\end{Highlighting}
\end{Shaded}

\begin{verbatim}
## 
## Call:
## lm(formula = Bloodpr ~ age, data = blood)
## 
## Residuals:
##     Min      1Q  Median      3Q     Max 
## -45.167 -15.583  -5.167  14.104  58.833 
## 
## Coefficients:
##             Estimate Std. Error t value Pr(>|t|)    
## (Intercept)  122.167      6.488  18.829  < 2e-16 ***
## age2          16.917      9.176   1.844  0.07423 .  
## age3          33.000      9.176   3.596  0.00104 ** 
## ---
## Signif. codes:  0 '***' 0.001 '**' 0.01 '*' 0.05 '.' 0.1 ' ' 1
## 
## Residual standard error: 22.48 on 33 degrees of freedom
## Multiple R-squared:  0.2816, Adjusted R-squared:  0.2381 
## F-statistic: 6.469 on 2 and 33 DF,  p-value: 0.004263
\end{verbatim}

\hypertarget{discussion-5}{%
\subsubsection{Discussion:}\label{discussion-5}}

\end{document}
